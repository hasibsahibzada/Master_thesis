
\begin{abstract}
\addchaptertocentry{\abstractname} % Add the abstract to the table of contents
Now a days, public displays are integrating more in urban environment, workplaces, supermarkets, bus/train stations, restaurants and more. These displays are vastly used as an advertising medium. Most advertisers use traditional advertising as their common driving business model, which passersby have no control over their contents, and these displays are often ignored because passersby expect uninteresting display contents, which is known as \emph{display Blindness}. On the other hand, a lot of researche about interactive advertisements in public displays have been going on that have optimism to boost advertisement effectiveness in the form of introducing new experience to passersby with the help of new sensing technologies. But up to my knowledge, no empirical research has been done to compare the effectiveness and behaviors of passersby on interactive and non-interactive advertisement in public displays.

This thesis followed the HCI and usability-engineering methods to choose, design, and develop three advertisements for \emph{Bauhaus-Walk}, which were non-interactive, body interactive and mobile interactive.  Each of them was deployed for one week in \emph{Weimar tourist information center}, and then the effectiveness of them were compared in between. Three measures of effectiveness were tested: The number of glances of passersby toward display, The number of \emph{Engaged} passersby and duration of their engagement. Beside that, the user behaviors were observed and among them two main behaviors of passersby were tested, which were the number of \emph{Honeypot effect} and \emph{Landing effect} toward display.

Results indicate that body interactive advertisement increased the attention level, the number of engagements and also the duration of engagement of passersby significantly compared to non-interactive advertisement. And along the effectiveness, the number of \emph{Landing} and \emph{Honeypot} effects were also improved. No one interacted with mobile interactive advertisement, and its attention level, number of engagement and users behaviors were not considerable compared to non-interactive advertisement. Based on the field observations of the display, which was situated at sideway, a new enhanced version of body interactive advertisement was developed to attract passersby from all display angles. The findings indicate that the enhanced body interactive advertisement significantly raised attention level and engagements than the previous body interaction, but both \emph{Landing} and \emph{Honeypot} effects were not remarkable.

I am optimistic that the future of advertising in public is tied with interactive displays. Researchers would use these methods and processes, which were followed in this thesis, to develop innovative interactive advertisement as their leading driving business models.


\end{abstract}
