% Chapte 9
\chapter{Conclusion} % Main chapter title

\label{Chapter10} % For referencing the chapter elsewhere, use \ref{Chapter1} 
\newpage

\section{Summary of research}

Public displays are in fact very complex research areas, and there are many reasons that could put the research at risk, for example, \emph{display locations}, \emph{Display sizes}, \emph{ Display orientation}, \emph{Physical setup} and more. Additionally there is no commonly accepted method for public display evaluation. \emph{Display locations:} the location of display also has influence on the attention level and motivations. If displays are located in front of the passers-by could have different attention level compared to at sideway, or if the display is placed in a higher compared to placed in lower place \cite{WhenPublicDisplays}. \emph{Display Size:} displays are found in various sizes and based on sizes they are used for various purposes, small sized displays are mostly used for single users and large displays would be suitable for more multi user interactive applications. The size can also influence on the attention level \cite{WhenPublicDisplays}. \emph{Display orientation:} orientation also influences the behavior of people in, how they are angled (horizontal or vertical). Each condition will produce different results\cite{DisplayAngleEffect}. \emph{Physical environment:} environment also has influence on attention and user behaviors. A display installed in a cafe or train station\cite{multimediaworkplace} would result in different outcomes compared to if installed in a library or workplace \cite{multimediaworkplace, semidisplay }. Because of all the issues, most researchers limit all their findings to a fixed environment and conditions, and can not generalize their study design and findings to whole the displays.  Therefore there is no commonly acceptable technique of evaluation for public displays \cite{HowToEvaluate}.

Consequently, Advertisings in public display also inherit the above issues in research field. Therefore in this study, specific conditions were taken in to account. The main study was conducted in \emph{Tourist information center}, and interactive and non-interactive advertisements were developed for \emph{Bauhaus-Walk}. The activating components of the \emph{conversion} (action) of the advertisements were measured and compared against each other. There are various activating components like \emph{emotions, motives, attentions and engagements} of passers-by, which would eventually lead to actions like participation in \emph{Bauhaus-Walk} program. The study compared the attention, engagement and other behaviors of the advertisement conditions and briefly answered the research questions below.





%Human emotions provides important feedbacks about a display, and there exists technologies that can track the human emotions and state of their conditions\cite{TrackeMotion}








\begin{itemize}

\item \textbf{ How can the attention of the passers-by be attracted?} (Chapter 3) \\
To design an attracting attention method for \emph{Bauhaus-Walk} interactive advertisements, an intense background study was done on attracting attention. Based on the reviews three attracting attention methods (moving eye, firework and silhouette representation) were developed. The number of \emph{Glances} of passersby were compared between these three methods and Non-interactive advertisement. Among the methods, only the silhouette representation significantly attracted more passersby. Additionally, this method is one of widely acceptable methods for interactive displays. As a result the silhouette representation was used for the rest of interactive advertisements.

\item \textbf{What is the attention level of passersby in interactive (body and mobile) and non-interactive advertisement?} (Chapter 8 and 9)\\
The body interactive advertisement had the highest amount of \emph{Glances} compared to the other advertisements. In Non-interactive advertisement 28.83\%, in body interactive 41.41\%, and in mobile version 33.76\% of the passers-by glanced toward the display. Enhanced body interactive advertisement strongly increased the number of \emph{Glances} by 51.11\% compared to Non-interactive and body interactive. The studies suggest that the interactive advertisements in all the conditions like body, mobile and enhanced body version, had higher attention level compared to non-interactive advertisement. 



\begin{table}[H]
\caption{Week sequence}
\label{tab:advertisementWeeks}
\centering
\resizebox{0.8\textwidth}{1.5cm}{  
\begin{tabular}{l | c | c }
\toprule
\tabhead{Advertisement} & \tabhead{Glanced passersby(\%)} & \tabhead{Ignored passersby} \\
\midrule
\textbf{Non-Interactive}     			&    28.83\%   &   71.17  \\
\textbf{Body Interactive}     			&    41.41\%   &   88.44  \\
\textbf{Mobile Interactive}  			&    33.76\%   &   66.24  \\
\textbf{Enhanced body Interactive}  	&    51.11\%   &   48.89  \\
\bottomrule
\end{tabular}
}
\end{table}






\iffalse
Honeypot effect, which increases the attention level of passers-by toward display, was also statistically increased with the use of body interactive advertisement compared to non-interactive advertisement, and the mobile interactive was insignificant to non-interactive. In enhanced advertisement version the honeypot effects were not increase compared to body interactive maybe because of limited number of days.

Landing effects that drags the passers-by suddenly toward display, in non-interactive advertisement this effect were less than interactive advertisement both body and mobile interactive. The landing effect in enhanced version was not significant because of its extended angle of tracking.
\fi



\item \textbf{How many passersby get engaged in interactive (body and mobile) and non-interactive advertisement?} \\
Involvement of passersby with the display, defines the effectiveness of advertisement. The involvement can be achieved if the passersby engage by watching the screen, reading or interacting with the advertisement display. In this study a person was marked as \emph{Engaged} if stood for more than 3 seconds in front of display.

In non-interactive advertisement people were reading or viewing the content of advertisement, and only 7.66\% of the whole passersby were engaged. The average duration for engagement was 34 seconds.

In body interactive advertisement, users were reading the content and at the same time they were motivated to play the game interaction using their body. From the whole passersby 11.56\% of them were engaged with the display. The average engagement duration was about 42 seconds, in which 19 seconds were spent in attraction/motivation part, 18 seconds in interaction and 4.6 seconds in advertisement video.

In mobile interactive advertisement, users were only reading or viewing the display with less interaction with their silhouette. But no interactions with the mobile devices were observed. Only 9\% of the passersby were engaged with engagement duration of 22 seconds in average.

In the enhanced advertisement version, passersby were also engaged with reading, playing with the silhouette and interacting with the game elements. 15\% passers-by were engaged with the average duration of 32 seconds. 

These findings recommend that enhanced and body advertisement engaged more participants than other advertisement techniques.



\begin{table}[H]
\caption{Week sequence}
\label{tab:advertisementWeeks}
\centering
\resizebox{0.8\textwidth}{1.5cm}{  
\begin{tabular}{l | c | c }
\toprule
\tabhead{Advertisement} & \tabhead{Engaged passersby(\%)} & \tabhead{Engagement duration} \\
\midrule
\textbf{Non-Interactive}     			&    7.66\%   &    34 sec  \\
\textbf{Body Interactive}     			&    11.56\%  &    42 sec  \\
\textbf{Mobile Interactive}  			&    9.0\%    &    22 sec  \\
\textbf{Enhanced body Interactive}  	&    15.0\%   &    32 sec  \\
\bottomrule
\end{tabular}
}
\end{table}






%\item Which of the advertisement types would increase the duration of engagement? \\

\item \textbf{What are passersby behaviors toward interactive (body and mobile) and non-interactive advertisements?} \\
In non-interactive advertisement, the behaviors of passersby toward display were more passive, calm and natural. Passersby selectively came near the display and used it as a source of information. The \emph{Display Blindness} was often observed, because a lot of passersby neglected the display. There was no influence of display on the environment around. Fewer \emph{Landing effects} and \emph{Honeypot effects} were also observered.

On the other hand, in the body interactive advertisement passersby were attracted quickly toward display. Passersby were curious about the display and they were waving hands or moving their body explicitly to learn about the interactivity. They felt the sense of joy and fun, and reacted according to the \emph{Call-to-Action}, they explored the interactions and played the game. The interactions were individuals and also in groups by calling other of their friends to join. Among 51 Engaged passersby, 17 of them ignored the advertisement video by leaving the display or standing one side until the video was over and start over the interaction. The dominance of the display over the area was felt if a person noticed the screen, the person had to leave the area or start the interaction. Higher \emph{Landing effects} and \emph{Honeypot effects} were also observed.

In mobile interactive advertisement, the passersby’s behaviors were like, being curious about their silhouette representation. They were waving their hands to confirm interactivity and coming closer to the screen to understand the interactivity of the system, but quickly left the display. Not interacting could be because of being skeptical about usage of phone in public, not understanding the connectivity to system, feeling unsecure, or feeling inappropriate interaction in that space. Fewer \emph{Landing effects} and \emph{Honeypot effects} were also observed.

In the extended advertisement version, the behaviors were very similar to body interactive advertisement. People felt the sense of joy, fun and play. 61 Passersby played the game and started exploring the locations. Group interactions and individual interactions were also seen. In this extended version people noticed the interactivity earlier and came toward display with very less landing effects. Side interactions were also observed, in which the people stood at side of the screen and were still playing with their silhouette. Less \emph{Landing effects} and \emph{Honeypot effects} were also observed.


 
\end{itemize}

\iffalse
\begin{itemize}
\item \textbf{R1: }What are the characteristics of a good and a bad Advertisement? (Chapter 3)  \\
From the short survey about characteristic of advertisement, participants answers were generalized and concluded that, a good advertisement from people points of view is an advertisement that provides most relevant content to the theme, environment and even passers-by, is short and precise, should have creativity and some kind of interactivity, and a bad advertisement is a kind of advertisement that does not suite to the passers-by and theme, is boring, has a lot of contents and annoys people by repetition, flashy texts and picture.
\item \textbf{R2: }Which method is better to attract passers-by's attention?  (Chapter 3) \\
Mirrored silhouette representation of passers-by attracted more participants than other abstract representation in a field study, which was conducted in university Mensa, and it is a preferred method in public displays in which passers-by feel comfortable and excitement. 

\item \textbf{R3: }How to create a suitable interactive and non-interactive advertisement?}(Chapter 4)\\
Any company that wants to advertise its product or service needs to define the advertising objectives in which it should prepare the content, scope, duration, and target a specific group of people. A good way to plan for the advertising goal is to conduct a focus group with the stakeholders. I planned two sessions for focus group with Bauhaus-Walk members, in the first session we discussed on content, theme, target audience and more. And on the second session proper prototype of interactive advertisement was discussed to finalize an advertisement that would be suitable for non-interactive, body interactive and mobile interactive. 

Interactive map was chosen for Bauhaus-walk, on which different interest points of Bauhaus was shown with picture and text description. Non-interactive advertisement would randomly explore the locations. In body interactive, passers-by would explore the locations by moving their body in front of the display. In Mobile interactive the same content would be explored by using a mobile controller. 

\item \textbf{R4: }How to design and evaluate Advertisement's Low-fidelity prototypes for public display?(Chapter 5) \\
When an advertisement is in its early stages, formative study would be a preferred way to evaluate the developed paper prototype, to be able to see most of the main functionalities and tackle the usability issues. Bauhaus-Walk advertisement’s interactive paper prototype was developed both for body interaction and mobile interaction, both prototypes could simulate almost the real life scenarios, a moderator was performing as a computer to get the user input and produce the output on the display, all the process was video recorded and had a follow up interview to get users opinions. 


\item \textbf{R5: }How to design and evaluate Advertisement's High-fidelity prototype for public display?
\item \textbf{R6: }What are the differences between non-interactive and interactive ad in public display?
\item \textbf{R7: }What could be enhanced to develop better advertisement in public display? 

\end{itemize}
\fi

\subsection{Advertisement development cycle}
The advertisement development cycle should mainly follow the advertising programs\cite{ad_def}, and be evaluated with the use of HCI methods. The advertising program is defined with series of steps to take. (1) \emph{Mission}, define the advertising objectives and goals. (2) \emph{Cost}, define the budgets for advertising location, medium, duration etc. (3) \emph{Message}, create advertising content, and evaluate the contents. (4) \emph{Media}, select an advertising medium for advertising campaign. And finally (5) \emph{Measurement}, to answer how the advertisement was effective. This thesis also partially followed this program for advertisement development and evaluations, which are discussed as below.

First and foremost, after many trials with university, I found \emph{Bauhaus-Walk} program that provides short tours for tourists in Weimar. This program became the advertiser and this was the start of communication process with them. By conducting \emph{Focus Groups} with \emph{Bauhaus-Walk} team members, I decided various things like, target group, location, duration of advertising, advertising message of \emph{Bauhaus-Walk} advertisement. The discussion was completed on two advertising prototypes and interactions techniques, which was body interactive and mobile interactive prototypes. These steps covered two essential programs (\emph{Mission, Message}.

Secondly, the \emph{Cost} was another issue, with which many things needed to be invested on like, (1)\emph{advertising location}, but with the support of \emph{Bauhaus-Walk} and Weimar tourist information center and university management, we could get the advertisement deployment for free which was for more than three weeks. Money needed to be spent on (2) \emph{Devices}, I needed a large LCD monitor, computers, Kinect cameras and other electronic devices for implementation. But I managed to get them from the University different departments. If the advertising were meant for long time and multiple locations, then it would have been expensive. 

Thirdly, the advertising prototypes (body and mobile) were evaluated using usability and HCI methods, in which the advertising message, interaction and usability issues were evaluated. This consisted two evaluation of, (1) Low-Fi prototype evaluation and (2) Hi-fi prototype evaluation. These evaluations were very helpful to decide for the correct \emph{Media, Message }. As a result three advertisements were developed which had the same content but different in interactivity, (1) Non-interactive, (2) Body interactive, and (3) mobile interactive.

Eventually, three of advertisements were deployed in Weimar Tourist information center each for one week. During those weeks different data gathering techniques were used like, direct observation, interviews of passers-by, depth recording and system logs. After that based on some observations on attention another extended advertisement application was developed and deployed again in tourist information, and followed the same data gathering techniques.  The new evaluation helped me to later assess the advertisement performance for each of the conditions.


\subsection{Advertisement performance}
The advertisement can perform better if the \emph{conversion rate} is higher. The \emph{Conversion rate} for \emph{Bauhaus-walk} advertisement would be that, how many people participated in the walk after the advertisement campaign. The comparison of interactive and non-interactive advertisement of \emph{Bauhaus-walk} was not to measure the final \emph{conversion rate}, because of many reasons. (1) There were already other existing advertising campaigns for Bauhaus-Walk. (2) The duration of advertisement was limited to five days each. (3) Limited reachability to wide range of people in city. (4) You may never know the reason of a person joining the walk; it could be because of interactivity of advertisement or because someone has told the person a month before the advertisement campaign even started. (5) Might be the people are motivated by the advertisement but does not have time to join this week and might join the other week.

Instead of measuring the \emph{conversion-rate}, the reasons that the conversion happens should be considered more. And if those reason are tackled then an effective and efficient advertisements can be developed. Those main reasons are the level of attention, motivations, involvement and emotions of people toward advertisement \cite{pervasiv_ad}, that can positively change people’s perception and attitude toward the product. This thesis compared these factors between Non-interactive, body interactive, mobile and extended body interactive advertisement, as discussed below.

\subsection{Attracting attraction}
Attracting attention of passers-by is the most crucial phase for the public display advertising, while most of the passers-by ignore the displays because of many reasons like. \emph{Information overload} \cite{Information_overload}, they think advertisements are irrelevant, boring and distractive to them \cite{banner_blindness, display_blindness}. There are two approaches of influencing the attention \emph{Top-down} and \emph{Bottom-up}, in top-down approach the passer-by has prior awareness of the display and change attention toward display, and in \emph{Bottom-up} the passer-by is unaware of the display and change attention toward display in case of an sudden external stimuli like color \cite{Luminance} or position \cite{capturingattention} change of an object in display.  

The \emph{Top-Down} approach cannot work for public display even if passers-by know about display because of \emph{display blindness}. Therefor \emph{Bottom-Up} approaches suites best in public display scenarios. Interactive advertising can use this approach and react based on passers-by and drag their attention toward it.



\begin{itemize}

\item \textbf{Silhouette representation} \\
Silhouette is a colored 2D shape of a person standing in front of a camera. Many researcher prefer the use of this representation in public display because of many reasons that are linked to attraction. (1) Sudden appearance of the silhouette when the person gets closer to display. (2) Color contrast of silhouette in relation to other silhouettes and background. (3) Responsiveness of silhouette.  So by combining all these elements this representation is the most attractive methods for body interactions \cite{LookingGlass}. The \emph{Bauhaus-Walk} interactive advertisement used this silhouette representation of passers-by to get their attentions the most.


\item \textbf{Extended silhouette representation} \\
This method was used to get passersby attention before they reach near the display. The method is using three cameras in the sides (right, left) and in the center to cover 180 degree in front of the display. This method increased the attention level dramatically then the previous method.


\end{itemize}

\subsection{Motivation}
To be motivated means \emph{to be moved to do something}\cite{motiv}. If someone does something like interacting with the display, it is because something else has moved or peacefully forced him to do so. Various forms of motivation exist and affects differently, which depends on person to person. Motivation is (1) Fun, (2) interesting, (3) captivating, (4) appealing, or has has (5) challenges, (6) fantasy or even (7) curiosity \cite{toward_motivation}. 

\begin{itemize}

\item \textbf{Silhouette representation:}\\
The silhouette representation was not just meant for attracting attention but also for motivation for many reasons. (a) It can become a fun and an interesting factor for people, because it is not a common thing to see a colored image instead of a full video image, and the different color of partner would become more interesting and playful. (b) It triggers curiosity among people and they would question that why are they shown in the display, or what is more to explore from the display. 

This representation was used in all interactive advertisement like in body, extended body and even mobile interaction techniques.

\item \textbf{Call-to-Action:} \\
Even if passers-by got motivated with silhouette representation, but they might still leave the display because they fear if they do something wrong or awkward, therefor to give them confidence and trust, call-to-action feature was developed. This is a responsive feature that follows with the silhouette of the person together and shows this text ``\emph{To play! Come near.}'', this text gives the user a goal for staying in front of display and at the same time it is a challenge for him/her to complete the task.

\end{itemize}


\subsection{Interaction}
When the passersby encounter with the interactive display for the first time, then there are many things that the display application should be ready for in terms of interaction. (1) Meaningful content \cite{ Meaningful_ad}, if the user does not feel comfortable with the content with which he/she is interacting, the user will ignore. (2) Meaningful interaction, it is not a good practice to create interactions that does not fit to the advertisement content and goal, and make the interactions in a way that makes the user feel shy or embarrassment in public \cite{EnticingPeople}. (3) The application usability,  The application should be able to provide an easy to use interaction, so that the users has full control and be able to complete the task, the use of right technology and technique is required to achieve this.

Based on many Focus group discussions and prototype (low-fi and high-fi) evaluation, I decided to create interactive advertisement for \emph{Bauhaus-Walk}. I developed two different interaction techniques, (a) body interaction and (b) mobile interaction, and both of them fit on the theme and goal of Bauhaus-Walk, the body \& mobile interaction techniques resemble the virtual walking of users in Weimar city and exploring Bauhaus locations, read chapter 7 for complete description of the interactions.


\subsection{Passers-by Behavior around display}
Passersby behave differently for certain types of display\cite{CylindricalScreen}   and certain physical setup or environment \cite{chained_displays, LookingGlass}. It is very crucial for advertisers to understand and know how the people react in front of their advertising displays, and that assists to develop an engaging and entertaining advertisement that could positively affect passersby perception and attitude toward advertiser’s products and services. In this thesis I compared the behaviors of people in front of non-interactive advertisement and interactive advertisement in Tourist information center. The below two effects were investigated in depth for all the advertisement conditions.

\subsubsection{Landing Effect}
In non-interactive and mobile interactive only 4 of engaged passersby performed  \emph{Landing effect}. Because passersby have not seen any obvious  relation to the screen, even in mobile the access information page shown on top of the silhouette. But in body interactive 12 (10\%) events of landing effects were observed which was 2 times higher than non-interactive. This is considered because of the clear silhouette representation and the \emph{Call-to-Action} text that made passersby to land back. The comparisons between the conditions were significant, and it concludes that silhouette can introduce more \emph{Landing effects}. On other hand the extended silhouette representation, which was in extended interactive advertisement, could not bring more \emph{Landing effects} (5 times), because people might have noticed themselves before they reach the end of display.

\subsubsection{Honeypot Effect}
The \emph{Honeypot effect} increases the attention level of the people toward the display. This effect in non-interactive display was weak (7 times) compared to body interactive display (15 times), which almost was double. The statistical analysis revealed that they are different significantly and it can conclude that the interactivity can increase the number of honeypot effects. But this effect was not statistically higher in extended body interaction (10 times) during 3 days, and it could be because of little number of days during the comparison, but it is certainly higher than non-interactive advertisement.



\subsection{Open Research Questions}


\subsubsection{Data gathering}

The data gathered in this thesis was more from direct and indirect observations, which involved manually going through all depth images frame by frame to refine the data logs for accurate measurement. It would be more effective to create advanced techniques for data gathering in public spaces in real time with high accuracy and less errors, there have been works on this as below.

\emph{Quantifying attention}, there are many applications that count glances, like \emph{IntraFace}\footnote{Intraface: \url{ http://www.humansensing.cs.cmu.edu/intraface/}, last accessed 12 jun 2016}\cite{Intraface} that can detect gaze direction by obtaining head orientation, extracting eye corners, detecting pupils and then finds out where is the person’s gaze direction, But this works only for in a controlled environment with multi users.  Another application was introduced in 2011 about the real world application of glance counting by F. De la Torre \cite{glancingcount} that could measure the person glance toward display, record passers-by speed and emotions. But I still personally doubt on these sort of applications in terms of accuracy and handling large amount of people glances and repeated glance of the same passer-by. With the use of new technologies in future these gaze detections would be more precise and stable.

It is easy to measure the \emph{duration of engagement} with interactive advertisement In online marketing, and is easy to measure how often the person interacts during day, week or month. It is easy because the individual computers can be traced using their IP or Mac addresses or even user accounts in social medias. But in public displays the situation seems difficult because unknown people pass by the display, and it is hard to keep track of one person during the day or week to measure the duration of engagement. There are applications \cite{glancingcount} using face detection algorithms that count the engagement durations but it is only for one time, the second time the person come is threated as a new person, or even there are a lot of false detection of faces and makes it hard to measure accurately. In future technologies I believe there will be more stable tracking applications to track users, but the display owners should deal with legal issues if people would allow to be tracked or not. 

\subsubsection{Impact of Advertisement types}
This thesis compared the non-interactive and interactive advertisement in terms of performance and effectiveness, but has not quantified the amount of impact of advertisement on passers-by, \emph{Attitude} toward a the product, or\emph{buying decision}. A very clear and precise evaluation techniques should be developed to assess the impacts of interactivity and non-interactivity of advertisement from various viewpoint and angles.

\subsubsection{Advertisement interaction design}
The \emph{Audience funnel}\cite{AudienceFunnel} interaction model consisted of 6 phases \emph{Passing by, Implicit interaction, Subtle interaction, direction interaction, multiple interaction and follow-up actions}. But in this thesis the body interaction model design had one extra phase called \emph{Ad video phase}, which was after the \emph{direct interaction} phase. The findings show that people did not stay longer than 4 seconds in average to watch the advertisement video. This feature was a disturbance for the people, who were interacting and were engaged with the display. This feature leads to a point that people would not be engaged anymore and future interaction designs should not include this phase in interaction design. A better solution would be to integrate short video clips inside the interaction phase, so that the people do not feel disconnected with the application and can still continue interaction and watch video.

\subsubsection{Enhanced silhouette color}
The extended version was in fact a good solution to attract attention in the side way public display. The integration of multiple Kinects extended the range, but all the passersby had the same color because of the lack of implementation. An algorithm is required to keep track of passers-by in all three cameras and give one consistent color to each individual. In this case every person will have their own dedicated color.

\subsubsection{Mobile interaction improvement}
The usability testing, which was conducted in the thesis showed that the mobile phone had performed extremely worse than body interaction technique. This is a major issue for public displays that use mobile phone interaction, and there is a need to research in this area from different perspectives like. (1) \emph{usability issues}, the usability issues are shared with mobile device and public display at the same time, there is a need to create usability evaluation techniques to tackle usability issues on both devices simultaneously. (2) \emph{Interaction technique}, create and develop interaction techniques that could be as easy as body interaction technique with less amount of cognitive load to user. And (3) \emph{Technology support}, the use of technology should be in a way that could support most famous platforms and at the same time it should be secure and reliable. 

As stated before, the finding in tourist information center shows that no one interacted with mobile devices during the five-day deployment. Some of the apparent issues regarding mobile interaction, were \emph{Physical setup, security, limited technology knowledge, lack of smartphones}, this raises questions that how to design a space for mobile interaction and increase trust between advertiser and passersby to allow them interact without any doubt. 





