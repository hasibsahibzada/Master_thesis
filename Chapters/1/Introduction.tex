% Chapter 1

\chapter{Introduction} % Main chapter title

\label{Chapter1} % For referencing the chapter elsewhere, use \ref{Chapter1} 
\newpage

%----------------------------------------------------------------------------------------

% Define some commands to keep the formatting separated from the content 
\newcommand{\keyword}[1]{\textbf{#1}}
\newcommand{\tabhead}[1]{\textbf{#1}}
\newcommand{\code}[1]{\texttt{#1}}
\newcommand{\file}[1]{\texttt{\bfseries#1}}
\newcommand{\option}[1]{\texttt{\itshape#1}}

%----------------------------------------------------------------------------------------
\section{Introduction}
From very old ages as from 13th century advertising has played an important role to promote customers attention and be able to compete with one another, The first paper advertisement was published at 1704 in an American newspaper called Boston News Letter, which was about houses and lands to be sold [16] and after that lots of business started to do their advertisements in newspapers, posters and banners. The first television ad was shown at 1941 on an American TV [15], this ad brought attention to a wide area of application and big business industries toward advertisement as a result the budgets raised much higher for advertisements and later advertisement entered the World Wide Web or so to say online advertising, which has evolved now to multi-billion dollar industry. Now because of the emerging new technologies and advancements, advertisements are in our smart phone applications, smart TV sets, tablet PCs and many other smart devices. And from past decades display screens are replacing print advertisements because of the easy reusability of the screen and convenient usage of them and providing dynamic contents.

Above all, still most of the advertisements are boring, time consuming, not clear for a lot of viewers, people tend to ignore advertisements because of many different reasons. Posters, banners and digital screens, which have static and dynamic contents respectively, are still done in a very bias way with considering the contexts, user?s interests, locations and many other factors. 

On the other hand, the use of technology with the advertisements could make the advertisement more attractive and interesting for viewers and open new ways and techniques to boost product purchases by customers, for example with the use of internet now more companies reserve spaces for their advertisement inside webpages by making flashy ads or playful interactive ads to attract users and redirect users to their webpages and so on. Even more interactive advertisements are now experienced in public spaces by allowing users to interact with their smartphone or do gestures or touch on large displays to perform easy tasks and get redirected to the shop or at least use them as a fun tool to remember about the product.

Additionally using body-sensing technologies, which are advancing day-by-day like Kinect Camera [17], could be used to allow passers-by to be engage without the use of other device, with which it would be easy for us to explore more possibilities of attraction methods, novel interactions and engagement techniques to provide to the users better experience and increase product purchase and interest.
%----------------------------------------------------------------------------------------

\section{Thesis Goal}
The research is concentrated in one particular environment, which is the Tourist Information Center, and currently there are more dynamic and static displays compared to interactive displays, people are a lot familiar with non-interactive displays and most of the times expect series of pictures and videos from these screens and threat it as a mean of advertisement, but there is a missing link between passers-by and advertisement, if this missing link gets connected somehow then advertisement would be more fun and engaging for people.

First, this thesis researches on advertisement in general to find out what are the people interest and expectation from a public display and how could the existing be changed in a way that people would like it and pay attention.

Second, this thesis researches on attraction attention level in public to find out which of the defined methods attracts passers-by attention toward the screen.

Third, it focuses on the missing link, that how to effectively connect people with advertisement so that people get attracted, motivated and get engaged with the screen.

Fourth, it conducts users studies and focus groups to make an advertisement that suites to the theme and is meaningful, from which two are interactive and one is non-active advertisement. Two of interactive advertisements consist of body interaction and mobile interaction and the non-active advertisement is same but the different phases are triggered automatically.

Additionally, conduct a comparative study on advertisements that first it would compare the non-interactive advertisement with interactive advertisement and second would compare two different interactive advertisements (Smartphone Vs. body) with each other.

And finally it proposes new attraction attention, motivation and engagement techniques for public displays and compares it with the previous interactive advertisement techniques.

\section{Research Questions}



\begin{table}[H]
\caption{Summary of Research Questions }
\label{tab:summaryofresearchquestions}
\resizebox{\textwidth}{!}{ 
\centering
\small
\begin{tabular}{ l  l  c}
\toprule
\tabhead{No.} & \tabhead{Research Questions}  & \tabhead{Chapter}\\
\midrule
R1   &  What are the characteristics of a good and a bad Advertisement?   	  	 & Chapter 3 \\
R2   &  Which method is better to attract passers-by's attention?  			  	 & Chapter 3 \\
R3   &  How to create a suitable interactive and non-interactive advertisement?  & Chapter 4 \\
R4   &  How to design and evaluate Low-fidelity prototypes for public display?   & Chapter 5 \\
R5   &  How to desgin and evaluate High-fidelity prototype for public display?   & Chapter 7 \\
R6   &  What are the differences between non-interactive and interactive ad?   	 & Chapter 8 \\
\bottomrule
\end{tabular}
}
\end{table}


%----------------------------------------------------------------------------------------

\section{Methodology}
Public displays are increasing day-by-day because the hardware prices are decreasing and people can afford to pay and install on at their shop, supermarket, tourist centers, libraries, museums and lot more places; therefor there is still not a single accepted guideline to design and evaluate public displays, This thesis is a qualitative research and used a user-centered design approach to carryout evaluation in different stages and most of the data gatherings were qualitative to take users feedbacks that led to the results and findings.


\subsection{Prototypes}
In this thesis prototypes were created in each stage like low-fidelity, high fidelity and the enhanced version of high fidelity prototypes, Each of the prototypes had their different versions and the latest versions were selected for the evaluation. There were lab prototypes and also on field prototypes, excessive efforts have been done to assure to make prototypes to be similar in various stages like low-fi and high-fi prototypes, and at the same time these prototypes should be robust and comply with technologies.  


\subsection{Evaluations}
Before even starting evaluations, many questions arise like where the location should be, what hardware shortcoming you have, and if you have other moderators to help you with the evaluation process and when they are free to assist you. During the thesis work different stages of evaluations have been completed like there were some evaluations that required only indoor in a controlled environment and some others required outdoor to get real data from public, The Low-fi and High-fi prototypes were evaluated in lab to do usability testing and do performance measuring, and the actual comparison of the advertisements (interactive and non-interactive) were done on field. 

The lab evaluations were fairly easily managed, but for the onsite field evaluations we had to deal with the many level of responsible personal to fix a date and location. 

During the evaluation process in public, privacy issues was an important factor that we had to be clear about and we tried to avoid taking pictures or video recordings unless by taking their permission and Kinect color silhouette recordings were used to hide identity of people.

Different methods of data gathering were used like interviewing people, taking onsite notes of the passers-by behavior, system logs and Kinect depth recording and some pictures. 


\subsection{Ethics}

\emph{Is this important, if yes which ethics should i mention?}


\section{Research context}
The research was carried out under Human Computer Interaction department in Bauhaus University Weimar over the course of one semester period the advertisement prototype was officially made for Bauhaus-Walk event and the main location that the comparison happened was in Weimar Tourist information center.



\section{Thesis outline}
\begin{itemize}

\item \textbf{Chapter 2:}
 This chapter discusses in-depth on various related issues like Advertisement, how it began, why is it influential, what is pervasive advertising, what are the common metaphors, in the second part of this chapter, it discusses on Public displays, the history of it, what are common technologies, what are they mostly used for, how engaging, attention and motivation methods are being used, what are the interaction techniques and how these displays could be evaluated.


\item \textbf{Chapter 3:}
This chapter focuses on advertisement to figure out what public really expect from advertisement in public displays and qualitatively summarizes good and bad advertisement, then this chapter discusses on various methods of attention level in public displays and proposes three different methods and evaluates them to chose the best one, the decision of this method will be used in further studies.


\item \textbf{Chapter 4:}
This chapter goes through the process that how and why the advertisement for Bauhaus-Walk was selected.


\item \textbf{Chapter 5:}
This chapter is the paper prototype evaluation, this chapter discusses on how the paper prototype was created for interactive advertisement public display and what were the results and findings from the participants. 


\item \textbf{Chapter 6:}
This chapter explains all the functionality and requirements of the applications, what technologies and hardware were being used and how to get the system running.


\item \textbf{chapter 7:}
This chapter conducts an advertisement high-fi evaluation and compares body interaction with mobile interaction techniques.


\item \textbf{chapter 8:}
This chapter makes the main goal of the thesis which is the comparison of non-interactive and interactive advertisement, the chapter explains about the study design along with data gathering techniques and how the data were evaluated and compared.


\item \textbf{Chapter 9:}
This chapter is an extension of the previous chapter and discusses the issues with the body interaction and how the body interaction could be enhanced to perform better in current existing public display setup, The chapter discusses on design study and how the experiment was conducted and how the results were compared with the older version of body interaction. 


\item \textbf{Chapter 10:}
Conclusion

\end{itemize}

