% Chapter 1

\chapter{Introduction} % Main chapter title

\label{Chapter1} % For referencing the chapter elsewhere, use \ref{Chapter1} 

%----------------------------------------------------------------------------------------

% Define some commands to keep the formatting separated from the content 
\newcommand{\keyword}[1]{\textbf{#1}}
\newcommand{\tabhead}[1]{\textbf{#1}}
\newcommand{\code}[1]{\texttt{#1}}
\newcommand{\file}[1]{\texttt{\bfseries#1}}
\newcommand{\option}[1]{\texttt{\itshape#1}}

%----------------------------------------------------------------------------------------

From very old ages as from 13th century advertising has played an important role to promote customers attention and be able to compete with one another, The first paper advertisement was published at 1704 in an American newspaper called Boston News Letter, which was about houses and lands to be sold [16] and after that lots of business started to do their advertisements in newspapers, posters and banners. The first television ad was shown at 1941 on an American TV [15], this ad brought attention to a wide area of application and big business industries toward advertisement as a result the budgets raised much higher for advertisements and later advertisement entered the World Wide Web or so to say online advertising, which has evolved now to multi-billion dollar industry. Now because of the emerging new technologies and advancements, advertisements are in our smart phone applications, smart TV sets, tablet PCs and many other smart devices. And from past decades display screens are replacing print advertisements because of the easy reusability of the screen and convenient usage of them and providing dynamic contents.

Above all, still most of the advertisements are boring, time consuming, not clear for a lot of viewers, people tend to ignore advertisements because of many different reasons. Posters, banners and digital screens, which have static and dynamic contents respectively, are still done in a very bias way with considering the contexts, user?s interests, locations and many other factors. 

On the other hand, the use of technology with the advertisements could make the advertisement more attractive and interesting for viewers and open new ways and techniques to boost product purchases by customers, for example with the use of internet now more companies reserve spaces for their advertisement inside webpages by making flashy ads or playful interactive ads to attract users and redirect users to their webpages and so on. Even more interactive advertisements are now experienced in public spaces by allowing users to interact with their smartphone or do gestures or touch on large displays to perform easy tasks and get redirected to the shop or at least use them as a fun tool to remember about the product.

Additionally using body-sensing technologies, which are advancing day-by-day like Kinect Camera [17], could be used to allow passers-by to be engage without the use of other device, with which it would be easy for us to explore more possibilities of attraction methods, novel interactions and engagement techniques to provide to the users better experience and increase product purchase and interest.
%----------------------------------------------------------------------------------------

\section{Thesis Goal}

Currently there are more dynamic and static displays compared to interactive displays. People are a lot familiar with non-interactive displays and most of the times expect series of pictures and videos from these screens and threat it as a mean of advertisement, but there is a missing link between passers-by and advertisement, if this missing link gets connected somehow then advertisement would be more fun for people. 

First, this thesis researches on advertisement in general to find out what are the people interest and expectation from a public display and how could the existing be changed in a way that people would like it and pay attention.

Second, it researches on attraction attention level in public to find out which of the defined methods attracts passers-by attention toward the screen.

Third, it focuses on the missing link, that how to effectively connect people with advertisement so that people get attracted, motivated and get engaged with the screen.

Fourth, it conducts users studies and focus groups to make an advertisement from which two are interactive and one is auto-active advertisement. Two of interactive advertisements consist of body interaction and mobile interaction and the auto-active advertisement is the same like above  but the different phases are triggered automatically. 

Additionally, conduct a comparative study on advertisements that first it would compare the traditional advertisement with interactive advertisement and second would examine two different interactive advertisements (Smartphone Vs. gesture) with each other.


And finally it proposes new attraction attention, motivation and engagement techniques for passers-by and compares the new effects with the previous interactive advertisement techniques.
%----------------------------------------------------------------------------------------

\section{Methodology}
\subsection{Prototypes}
\subsection{Evaluations}
\subsection{Ethics}

\section{Summary of thesis contribution}
\section{Research context}
\section{Thesis outline}

\section{In Closing}

You have reached the end of this mini-guide. You can now rename or overwrite this pdf file and begin writing your own \file{Chapter1.tex} and the rest of your thesis. The easy work of setting up the structure and framework has been taken care of for you. It's now your job to fill it out!

Good luck and have lots of fun!

\begin{flushright}
Guide written by ---\\
Sunil Patel: \href{http://www.sunilpatel.co.uk}{www.sunilpatel.co.uk}\\
Vel: \href{http://www.LaTeXTemplates.com}{LaTeXTemplates.com}
\end{flushright}
