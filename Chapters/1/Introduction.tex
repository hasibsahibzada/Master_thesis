% Chapter 1

\chapter{Introduction} % Main chapter title

\label{Chapter1} % For referencing the chapter elsewhere, use \ref{Chapter1} 
\newpage



%----------------------------------------------------------------------------------------
\section{Introduction}
Advertisement is the mean of conveying message(s) to people about something from which both producers and consumers get benefit, as P. Kotler \cite{ad_def} defines advertisement as ``\emph{any paid form of non-personal presentation and promotion of ideas, goods, or services by an identified sponsor. Advertisers include not only business firms but also charitable, nonprofit, and government agencies}''. Technology is dramatically changing our lives and it is integrating in our environment and obviously it is affecting the advertisement too, with the use of media, advertisements are published in TV, newspapers, radio, magazines, banners, mobile phones, public displays and more and currently advertisements are found in the form of, (1) Non-Interactive advertisement and (2) Interactive Advertisement.

Non-interactive advertisement is the traditional advertisement that “the presentation of content is linear and the consumer is passively exposed to product information” \cite{Non_inter_vs_interAd}, user has no control over the flow of the advertisement. It is delivered using media like TV, radio, public displays, banners and many other various mediums. Above all, still most of these advertisements are boring, not clear for a lot of viewers, people tend to ignore advertisements \cite{display_blindness, banner_blindness}

Where on the other hand, with the use of innovative technologies, advertisers can make interactive advertisement, which can be more attractive and interesting and open new ways and techniques to boost advertisement effectiveness \cite{add_effectivenss}, Interactive advertisement is a type of advertisement that is done by using various interactive media like Internet, mobile phones and public displays, and it allows users to actively traverse the advertisement content and depends on where the user want to go from one step to another \cite{Non_inter_vs_interAd}. Advertisers reserve appropriate website sections for their interactive advertisements and the use of interactive public displays are increasing to provide passers-by opportunity to interact with advertisement contents, for example using smartphone to control interactive elements or by using body-sensing technologies, like Kinect\footnote{Microsoft Kinect: \url{https://developer.microsoft.com/de-de/windows/kinect},Last accessed: 1/05/2016 at 13:21:00} cameras, which could be used to allow passers-by to be engaged without the use of any other device, these technologies with which it would be easy for us to explore more possibilities of attraction methods, novel interactions and engagement techniques and provide the users with better experience and increase their interest. 

There is a need to investigate that how much interactive advertisement in public displays are attractive, engaging and can change user behaviors compared to non-interactive advertisement, if they are significantly different, what kind of models and interactive design space would be suitable for future interactive advertisings to improve audience attention level and engagement experience. Furthermore, this thesis explores and investigates public display advertisements in general, what makes a suitable advertisement for audience, what are the common attraction attention methods, is there a difference in body interactive advertisement and mobile interactive advertisement and what kind of environmental setup is required.

In order to be able to conduct the advertisement research, there was a need to create realistic advertisement and realistic target groups and environment, therefor at the beginning for attracting attention application’s evaluation University Mensa\footnote{Bauhaus University Mensa: http://www.stw-thueringen.de/english/dining-halls/facilities/weimar/mensa-am-park.html, last accessed 25 may 2016} was chosen and for the advertisement’s content \emph{Bauhaus-Walk} \footnote{Bauhaus Walk: https://www.uni-weimar.de/en/university/profile/bauhausatelier/bauhaus-walk/, last accessed 25th May 2016} was chosen to make advertisement for and through Bauhaus-Walk members \emph{Weimar Tourist Information Center} \footnote{Weimar Tourist Information Center: \url{http://www.weimar.de/homepage/}, last accessed 10 April 2016} was contacted to install the advertisement display and evaluate our applications in wild.

%As Norman \cite{norman} describes that there are three different level of interactive computer system, \emph{visceral, behavioral} and \emph{reflective}, visceral level is about the first impact or impression of a product it is about its appearance and look, behavioral level is about the use and experience with something, and finally the reflective level which is the highest level of feeling, emotions and thoughts on something. Taking these levels in consideration Non-interactive advertisement can reach only the first visceral level and cannot go further but Interactive advertisement can reach the other two behavioral and reflective levels too, and can build strong experience and impressive effects, if the advertisement is created in a innovative and fun form, more audience would likely pay significant attention to the content, which would consequently equate to higher advertisement recalls\cite{add_effectivenss}, and would increase involvement of both users and product that is believed to have an effective advertising to convey message\cite{audience_involvement}.

%Public displays are increasing because they are becoming very cheap and common and are found almost everywhere in different sizes and features; these displays are used as information or advertisement medium, many researches have been done on how to make these displays more attractive \cite{DesignSpace, attention1}, and engaging \cite{toward_motivation, pervasiv_ad} and at the same time various audience behaviors have been studied around display\cite{hole_space, EnticingPeople} to observe user attitude toward these situated displays, J. Müller \cite{intro_to_pervasive_ad} introduced ''\emph{Pervasive Advertising}'' which is made by the use of pervasive technologies\cite{pervasiv_ubiquitous} that enables quality interactions, audience measurement and personalization so easy and this could be the future of advertising that most industries would adapt.


%----------------------------------------------------------------------------------------

\section{Advertisement performance}
When a company develops an advertisement and campaigns it for long time in different locations, mainly expects to have a higher \emph{conversion rate}, Conversion rate is ``\emph{The percentage of visitors who take a desired action.}''\cite{convrate} there are different forms of action goals, like it could be buying the product, joining an event, registering for a website, paying a charity or even could be participation in a rally or protest, so it really depends on what is the main goal behind a particular advertisement, and the conversion rate is measured by the number of people who performed the action divided by total number of visitors. Occasionally conversion rate is measured in Internet advertising with various metrics like, CPM, CPC, CPA and more, which are discussed in detail in background chapter, and to understand the motive behind the conversion like what made them converted is an important question to ask, if we tackle those questions then we can create effective and efficient advertisements, those main reasons are the level of attention, motivations, involvement and emotions of people with the advertising product \cite{pervasiv_ad}.

\emph{Attention:}, ``\emph{Attention is the process that, at a given moment, enhances some information and inhibits other information. The enhancement enables us to select some information for further processing, and the inhibition enables us to set some information aside.}''\cite{Attention}, Higher attention would increase the high recall of advertisement too \cite{add_effectivenss}, attention is the first phase that can take user to be involved.

\emph{Motivation:}
To be motivated means \emph{to be moved to do something}\cite{motiv}.The motivation is an important thing after a person has been attracted toward display; the motivation can be achieved by making passers-by curious about the screen, challenging them by a game or bring some sort of fantasy in application. In the design of body and mobile interaction models, the above factors were taken in mind for motivation and two features were implemented as described bellow.\cite{ toward_motivation}. 

\emph{Involvement:} Involvement describes the relationship of audience to a product and the strength can define effectiveness of the advertisement and engagement is a form of involvement with the product. Technologies are there that can measure involvement like the attention level or duration of interaction with a product.

\emph{Emotion:}, ``\emph{Emotions is an affective state of consciousness in which joy, sorrow, fear, hate, or the like, is experienced, as distinguished from cognitive and volitional states of consciousness.}''\cite{emotiondef}, these emotions always can influence users to change their attitude and how they think about a product or service and by tracking user emotions advertisement could be adjusted in real time.




\section{Research Questions}
The \emph{Conversion rate} for Bauhaus-walk advertisement would be that, how many people participated in the walk after the advertisement campaign, but in this thesis, I do not measure the conversion rate, because it is possible that people maybe converted from other unknown reasons like a friend might tell or existing advertising campaign. To measure conversion-rate the only solution is to take small interviews of each individual, who joint the walk and question the reason of joining, which is time consuming process and should be continued for long time to track all the person, who were exactly affected by one of advertisement campaign.

Because of the reasons mentioned in advertisement performance section like, attention, motivation, engagement, and emotions, that influence the \emph{conversion-rate} of an advertisement, therefor it would be more appropriate to compares these important aspects between interactive and non-interactive advertisement, rather than comparing the \emph{conversion-rate} itself. The bellow lists the main research questions that need to be find out for interactive and non-interactive advertisement.


\begin{itemize}
\item Which method is better to attract passers-by's attention? 
\item How is the attention level in interactive (body and mobile) and non-interactive advertisement?
\item How many passers-by are engaged in interactive (body and mobile) and non-interactive advertisement?
\item What are passers-by behavior toward interactive (body and mobile) and non-interactive advertisemen?
\end{itemize}

\section{Procedures}

The main purpose of the thesis is the comparison of Non-interactive and Interactive advertisement in the domain of attracting attention, engagement and passers-by behaviors, but it would have not been compared unless the well functional and meaningful advertisement applications were not developed and evaluated. 

Therefore, first, this thesis researches on advertisement in general to find out what are the people interests and expectations from public display and how could the existing advertisement be changed in a way that people would like it and pay attention.

Second, it investigates on attraction attention phase for public display advertisement to find out which of suitable methods attract passers-by attention toward the screen.

Third, it conducts user studies and focus groups to find out what make suitable advertisements that fits \emph{Bauhaus-Walk} theme, in which two are interactive and one is non-active advertisement. Two of interactive advertisements consist of body interaction and mobile interaction.

Fourth, it evaluates the low-fidelity and high fidelity of the interactive advertisement applications (mobile and body) and explores that which of these interactive modalities perform better and how the participants give feedback about their usage in public space.

Finally, it conducts a comparative study on non-interactive advertisement with interactive advertisement (body and phone), which was installed in tourist Information center location, to find out which of them attracted the most passers-by, how many of the users were engaged and how their behavior was in relation to the display.

And based on the result and findings, it proposes new enhanced interactive advertisement technique in the context of public displays and compares it with the previous advertisements techniques.



\iffalse
\begin{table}[H]
\caption{Summary of Research Questions }
\label{tab:summaryofresearchquestions}
\resizebox{\textwidth}{!}{ 
\centering
\small
\begin{tabular}{ l  l  c}
\toprule
\tabhead{No.} & \tabhead{Research Questions}  & \tabhead{Chapter}\\
\midrule
R1   &  What are the characteristics of a good and a bad Advertisement?   	  	 & Chapter 3 \\
R2   &  Which method is better to attract passers-by's attention?  			  	 & Chapter 3 \\
R3   &  How to create a suitable interactive and non-interactive advertisement?  & Chapter 4 \\
R4   &  How to design and evaluate Advertisement's Low-fidelity prototypes for public display?   & Chapter 5 \\
R5   &  How to design and evaluate Advertisement's High-fidelity prototype for public display?   & Chapter 7 \\
R6   &  What are the differences between non-interactive and interactive ad in public display?   	 & Chapter 8 \\
R7   &  What could be enhanced to develop better advertisement in public display?   	 & Chapter 9 \\
\bottomrule
\end{tabular}
}
\end{table}
\fi








%----------------------------------------------------------------------------------------

\section{Methodology}
Public displays are very complicated medium for advertisings, but the fact is that they are replacing traditional paper based advertisements, This thesis was mostly based on qualitative research and uses a user-centered design approach to carryout evaluation in different stages of prototype and for advertisement comparison quantitative statistical analysis tools were used to compare the performance of them.
\hilight{describe more if required}


\subsection{Prototypes}
In this thesis prototypes were created in each stage like low-fidelity, high fidelity and the enhanced version of high fidelity prototypes, Each of the prototypes had their different versions and the latest versions were selected for the evaluation. There were lab prototypes and also on field prototypes, excessive efforts have been done to assure to make prototypes to be similar in various stages like low-fi and high-fi prototypes, and at the same time these prototypes should be robust and comply with technologies.  


\subsection{Evaluations}
Before even starting evaluations, many questions arise like where the location should be, what hardware shortcoming you have, and if you have other moderators to help you with the evaluation process and when they are free to assist you. During the thesis work different stages of evaluations have been completed like there were some evaluations that required only indoor in a controlled environment and some others required outdoor to get real data from public, The Low-fi and High-fi prototypes were evaluated in lab to do usability testing and do performance measuring, and the actual comparison of the advertisements (interactive and non-interactive) were done on field. 

The lab evaluations were fairly easily managed, but for the onsite field evaluations we had to deal with the many level of responsible personal to fix a date and location. 

During the evaluation process in public, privacy issues was an important factor that we had to be clear about and we tried to avoid taking pictures or video recordings unless by taking their permission and Kinect color silhouette recordings were used to hide identity of people.

Different methods of data gathering were used like interviewing people, taking onsite notes of the passers-by behavior, system logs and Kinect depth recording and some pictures. 


\section{Research context}
The research was carried out under Human Computer Interaction department in Bauhaus University Weimar over the course of one semester period the advertisement prototype was officially made for \emph{Bauhaus-Walk} event and the main location that the comparison happened was in Weimar Tourist information center.
\hilight{describe more if required}

\newpage
\section{Thesis outline}
To make the thesis document more readable and understandable for the readers, it is divided in to five parts, and each of these sections contain various chapters


\begin{table}[H]
\centering
\caption{Thesis Outline}
\label{Thesis_Outline}
\begin{tabular}{|c|l|}
\hline
Sections                                      & Chapters                                                                                                    \\ \hline
\multirow{2}{*}{Introduction}                 & \#1: Introduction                                                                                           \\ \cline{2-2} 
                                              & \#2: Background                                                                                             \\ \hline
\multirow{5}{*}{Pre Advertisement Comparison} & \#3: Attention Attraction                                                                                   \\ \cline{2-2} 
                                              & \#4: Advertisement Decision                                                                                 \\ \cline{2-2} 
                                              & \#5: Advertisement Low-Fi evaluation                                                                        \\ \cline{2-2} 
                                              & \#6: Advertisement Development                                                                              \\ \cline{2-2} 
                                              & \#7: Advertisement High-Fi Evaluation                                                                       \\ \hline
\multirow{2}{*}{Advertisement Comparison}     & \begin{tabular}[c]{@{}l@{}}\#8: Comparison of Interactive and \\ Non-Interactive Advertisement\end{tabular} \\ \cline{2-2} 
                                              & \begin{tabular}[c]{@{}l@{}}\#9: Design and evaluation of  \\ ExtendedInteractive Advertisement\end{tabular} \\ \hline
\multirow{3}{*}{Conclusion And Appendices}    & \#10: Conclusion                                                                                            \\ \cline{2-2} 
                                              & References                                                                                                  \\ \cline{2-2} 
                                              & Appendices                                                                                                  \\ \hline
\end{tabular}
\end{table}



\begin{itemize}
\item \textbf{Chapter 2:}
 This chapter discusses in-depth on various related issues like Advertisement, how it began, why is it influential, what is pervasive advertising, what are the common metaphors, in the second part of this chapter, it discusses on Public displays, the history of it, what are common technologies, what are they mostly used for, how engaging, attention and motivation methods are being used, what are the interaction techniques and how these displays could be evaluated.


\item \textbf{Chapter 3:}
This chapter focuses on advertisement to figure out what public really expect from advertisement in public displays and qualitatively summarizes good and bad advertisement, then this chapter discusses on various methods of attention level in public displays and proposes three different methods and evaluates them to chose the best one, the decision of this method will be used in further studies.


\item \textbf{Chapter 4:}
This chapter goes through the process that how and why the advertisement for Bauhaus-Walk was selected.


\item \textbf{Chapter 5:}
This chapter is the paper prototype evaluation, this chapter discusses on how the paper prototype was created for interactive advertisement public display and what were the results and findings from the participants. 


\item \textbf{Chapter 6:}
This chapter explains all the functionality and requirements of the applications, what technologies and hardware were being used and how to get the system running.


\item \textbf{chapter 7:}
This chapter conducts an advertisement high-fi evaluation and compares body interaction with mobile interaction techniques.


\item \textbf{chapter 8:}
This chapter makes the main goal of the thesis which is the comparison of non-interactive and interactive advertisement, the chapter explains about the study design along with data gathering techniques and how the data were evaluated and compared.


\item \textbf{Chapter 9:}
This chapter is an extension of the previous chapter and discusses the issues with the body interaction and how the body interaction could be enhanced to perform better in current existing public display setup, The chapter discusses on design study and how the experiment was conducted and how the results were compared with the older version of body interaction. 


\item \textbf{Chapter 10:}
Conclusion

\end{itemize}

